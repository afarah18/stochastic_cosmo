\documentclass[]{article}
\usepackage{amsmath}
\usepackage[utf8]{inputenc}


\newcommand{\Ogw}{\Omega_{\mathrm{gw}}}
\newcommand{\rhor}{r_{\text{hor}}}

%opening
\title{Notes on: \emph{Limits on the Size of the Universe with the Stochastic GW Background}}
\author{Amanda Farah}

\begin{document}

\maketitle

%\begin{abstract}
%Here, Amanda pretends to be a cosmologist. She's not good at it so she has to go back to the 19th century. 
%
%\end{abstract}

\section{Introduction}
We aim to set a limit on the size of various simple universes by using the nondetection of a stochastic background of gravitational waves. 
We assume that the stochastic background is dominated by gravitational waves (GWs) from compact binary coalescences (CBCs) and assume a constant rate of CBCs throughout the universe.

\section{Calculation}

We start with Equation 5 in \cite{Phinney_theorem}. Namely,

\begin{equation}
	\rho_c c^2 \Ogw(f) =  \int_{0}^{\infty} \frac{N(z)}{1+z} (f_r \frac{dE_{\mathrm{gw}}}{df_r})_{f_r = f(1+z)} \mathrm{d}z ,
	\label{eq:phinney_theorem}
\end{equation}
where $\frac{dE_{\mathrm{gw}}}{df}$ is the energy spectrum emitted by a CBC source, $f_r$ is the frequency in the source frame, and $N(z)$ is the comoving number density of sources per redshift interval.
The factor of $(1+z)$ in the denominator is to account for redshifting of the gravitons since emission. 
$\rho_c = 3 H_0^2/(8\pi G)$ is the critical density to close the universe.
	
\subsection{Minkowski Universe}	
We wish to consider the case of a static, Minkowski spacetime and constrain the size of this universe under the assumption that the number density of stochastic sources is a spatial and temporal constant throughout the universe.
In this scenario, redshift has no meaning, so we must convert the integral over redshift in Equation~\ref{eq:phinney_theorem} to an integral over distance.
To do this, we assert that the number of galaxies should be invaraiant under a parametrization in terms of redshift or time. 
That is,

\begin{align}
\# = (\frac{dN}{dV_C dz}) dV_C dz  &= (\frac{dN}{dV_C dt}) dV_C dt \nonumber \\
\frac{dN}{dV_C dz} &= (\frac{dN}{dV_C dz}) \frac{dt}{dz}
\label{eq:num_per_vt}
\end{align} 
where ``C" indicates a comoving value.

Using $N(z) = \frac{dN}{dV_C dz}$ and substituting Equation~\ref{eq:num_per_vt} into Equation \ref{eq:phinney_theorem}, we have
\begin{align*}
\rho_c c^2 \Ogw(f) &=  \int_{0}^{\infty} \frac{dN}{dV_C dt} \frac{dt}{dz} \frac{1}{1+z} (f_r \frac{dE_{\mathrm{gw}}}{df_r})_{f_r = f(1+z)} \mathrm{d}z \\
\rho_c c^2 \Ogw(f) &=  \int_{0}^{\infty}\frac{dN}{dV_C dt} \frac{1}{1+z} (f_r \frac{dE_{\mathrm{gw}}}{df_r})_{f_r = f(1+z)} \mathrm{d}t 
\end{align*}

We wish here to use constraints on $\Ogw$ to put an upper limit on the horizon distance $\rhor$ of a static Minkowski universe, so we change the upper limit of the integral from $\infty$ to $\rhor$. 
Noting that in a non-expanding universe, gravitons do not get cosmologically redshifted, we set z=0. 
This means that the frequency and energy in the source frame ($r$) are the same as in the detector frame.
Thus, we have 
\begin{equation*}
	\rho_c c^2 \Ogw(f) =  \int_{0}^{\rhor} \frac{dN}{dV_C dt} \frac{dE_{\mathrm{gw}}}{df} \mathrm{d}t .
\end{equation*}

We can turn this into an integral over distance along our past light cone by noting that gravitational waves travel at the speed of light $c$:
\begin{equation}
	\Ogw(f) =  \frac{1}{\rho_c c^3} \int_{0}^{\rhor} \frac{dN}{dV_C dt} \frac{dE_{\mathrm{gw}}}{df} \mathrm{d}r
	\label{eq:euclid_omega}
\end{equation}

Evaluating with the assumption that $\frac{dN}{dV_C dt}$ is a spatial constant, we have
\begin{align*}
	\Ogw(f) &= \frac{dN}{dV_C dt} \frac{f}{\rho_c c^3} \frac{dE_{\mathrm{gw}}}{df} \int_{0}^{\rhor} dr \\
	&= \frac{dN}{dV_C dt} \frac{f}{\rho_c c^3} \frac{dE_{\mathrm{gw}}}{df} \rhor  .
\end{align*}
Note that $\frac{dN}{dV_C dt}$ is a merger rate, so we will call it $\mathcal{R}$ for simplicity. 
For CBC sources, $\frac{dE_{\mathrm{gw}}}{df} = \frac{\mathcal{M}^{5/3}(G\pi)^{2/3}}{3 f^{1/3}}$ so
\begin{equation*}
	\Ogw(f) = \frac{\mathcal{R} f^{2/3}}{3 \rho_c c^3} \mathcal{M}^{5/3}(G\pi)^{2/3} r_{\text{hor}} .
\end{equation*}
Rearranging and substituting $\rho_c = 3 H_0^2/(8\pi G)$,
\begin{equation*}
	r_{\text{hor}} = \frac{9\Ogw(f) H_0^2 c^3}{8 \mathcal{R}} (f)^{-2/3} ( G \pi \mathcal{M})^{-5/3} .
\end{equation*}

In the GW stochastic background literature (e.g. \cite{O1_stoch,O2_stoch}) $\Ogw$ for a CBC source is commonly written as $\Ogw = \Omega_{\text{ref}} (\frac{f}{f_{\text{ref}}})^{2/3}$ so that $\Omega$ can be calculated at a single frequency. 
The reference frequency for LIGO searches is $f_{\text{ref}} = 25 Hz$.  

Making this replacement we have
\begin{equation}
	\rhor=\frac{9\Omega_{\text{ref}} H_0^2 c^3}{8 \mathcal{R}} (f_{\text{ref}})^{-2/3} ( G \pi \mathcal{M})^{-5/3} .
	\label{eq:final_expression}
\end{equation}
This is our final expression.
Now comes the time to compute with characteristic values, which we summarize in Table \ref{tab:vals}. 
The rate is taken to be the local value computed in LIGO's second observing run (\cite{O2_rates}) and the upper limit on $\Omega_{\text{ref}}$ is taken to be the one reported in the isotropic search from LIGO's first and second observing runs (\cite{O2_stoch}).
The chirp mass $\mathcal{M}$ is calculated for the case of two 30 $M_{\bigodot}$ black holes.
\begin{table}
	\centering
\begin{tabular}{c|c|c|c}
	$\mathcal{R} (\mathrm{yr}^{-1} \mathrm{Gpc}^{-3})$ & $\mathcal{M} (M_{\bigodot})$  & $\Omega_{\text{ref}} $ & $H_0 (km/s/Mpc)$ \\
	\hline
	 53.2 & $26$ & $4.8\times10^{-8}$ & 67.9 \\
\end{tabular} 
\caption{List of parameters used}
\label{tab:vals}
\end{table}

Using these parameters, we get an upper limit on the horizon distance $\boxed{r_{\text{hor}} \leq 19.7 \mathrm{Gpc}}$

\subsubsection{Considering Distributions of Source Parameters}
Up until now, we have only considered chirp masses from a single source type.
If instead we would like to consider a full distribution of source types, $N(z) (f_r \frac{dE_{\mathrm{gw}}}{df_r})$ in Equation~\ref{eq:phinney_theorem}, $N(z) (f_r \frac{dE_{\mathrm{gw}}}{df_r})$ is replaced by $\sum_{i} N_i (z) (f_r \frac{dE_{\mathrm{gw},i}}{df_r})$, where the sum is over source types $i$. 
For continuous values of source parameters, Equation~\ref{eq:euclid_omega} becomes
\begin{equation*}
\rho_c c^2 \Ogw(f) =  \int_{0}^{\rhor} \frac{H_0}{c} \int d\vec{\theta} N(D_C,\vec{\theta}) (f \frac{dE(f,\vec{\theta})_{\mathrm{gw}}}{df}) \mathrm{d}D_C ,
\end{equation*}
where $\vec{\theta}$ are the source parameters. 
Here, we are primarily concerned with the mass distribution of sources, so we only consider $\vec{\theta} = \mathcal{M}$.
As before, we substitute $\frac{dE_{\mathrm{gw}}}{df} = \frac{\mathcal{M}^{5/3}(G\pi)^{2/3}}{3 f^{1/3}}$ and assume $N$ is constant in comoving volume. 
This yields
\begin{equation}
\rho_c c^2 \Ogw(f) =  \rhor \frac{H_0}{c} \frac{(f G \pi)^{2/3}}{3} \int d\mathcal{M} N(\mathcal{M}) \mathcal{M}^{5/3}  .
\label{eq:Ogw_with_N_of_M}
\end{equation}
Now all that is left to be done is find an expression for $N(\mathcal{M})$ for a given mass distribution and rate.

\begin{align*}
 N(\mathcal{M}) &= \int d\mathcal{M} \frac{d N(\mathcal{M})}{d\mathcal{M}} \\
  &= \int d\mathcal{M} \int dt \frac{d\mathcal{R}}{d\mathcal{M}}.\\
\end{align*}

For simplicity, we will assume all systems have equal mass components and then re-write in terms of an individual component mass, $m_1$.
In this scenario, $\frac{d\mathcal{R}}{d\mathcal{M}} = 2^{1/5} \frac{d\mathcal{R}}{d m_1}$ and Equation~\ref{eq:Ogw_with_N_of_M} becomes
\begin{equation}
\rho_c c^2 \Ogw(f) =  \rhor \frac{H_0}{c} \frac{(f G \pi)^{2/3}}{3 (2)^{1/3}} \int d t \int d m_1 \frac{d\mathcal{R}}{dm_1} m_1^{5/3} .
\label{eq:Ogw_with_R_of_M}
\end{equation}

We take the mass distribution model $\frac{d\mathcal{R}}{dm_1}$ from Equation 17 of~\cite{O2_rates} and modify it to get rid of redshift evolution, obtaining
\begin{equation*}
\frac{d\mathcal{R}}{dm_1 dm_2} = \mathcal{R}_0 p(m_1, m_2 | \gamma, m_{min},m_{max})
\end{equation*}
where $\gamma, m_{min}, m_{max}$ are the power law slope, minimum and maximum possible component masses, respectively, and are determined by the model fit.
We combine this with the model from~\cite{picky_partners} (namely Equation 5) and choose a pairing function that will enforce equal masses to get
\begin{equation*}
\int d m_1 \frac{d\mathcal{R}}{dm_1} m_1 ^{5/3} = \frac{\mathcal{R}_0 (2\gamma +1)}{m_{max}^{2\gamma +1} - m_{min}^{2\gamma +1}} [\frac{m_{max}^{2\gamma +8/3} - m_{min}^{2\gamma +8/3}}{(2\gamma +8/3)}] . 
\end{equation*}
Plugging this into Equation~\ref{eq:Ogw_with_R_of_M}, using the values listed in Table~\ref{tab:vals}, and rearranging yields $\boxed{\rhor \leq 39.2 \mathrm{Gpc}}$. 

\subsection{FLRW Universe}
Motivated by Table 1 in \cite{Wesson_classical_olbers}, we note that different cosmological parameters in an FLRW universe predict different values of the stochastic GW background. 
We therefore wish to see if it is possible to constrain these parameters with the current limits on $\Ogw$. 

Starting again from Equation~\ref{eq:phinney_theorem} and substituting in 

\section{Issues}
The rate and $\Omega_{\text{ref}}$ are taken from LIGO analyses which assume an FLRW universe, so using those values might not be consistent with this analysis. 
However, the rate measurement is done locally, in which the universe is basically Minkowski, so hopefully it is not a large effect. 
Additionally, $\Ogw$ is measured based on the strain signal alone - I think it might not have a cosmology folded in.\\

The number density N is calculated by multiplying the rate by the amount of time it takes GWs to reach us from the horizon distance (namely, $N = \mathcal{R} r_{\text{hor}}/c$). Presumably, the rate at each point should be multiplied by the time it takes GWs to reach us from the radius at that point, or some other time that I just can't think of at this time.\\

If instead the time used is the time spent observing the background $t_{\rm obs}$, Equation~\ref{eq:final_expression} instead becomes
\begin{equation}
r_{\text{hor}}=\frac{9\Omega_{\text{ref}} H_0 c^3}{8 \mathcal{R} t_{\rm obs}} (f_{\text{ref}})^{-2/3} ( G \pi \mathcal{M})^{-5/3} .
\end{equation}
Using $t_{obs} = 1.5$ yr gives $r_{\text{hor}} = 8.4 \times 10^{11}$Gpc, which is very large and either indicates an issue with a Euclidian universe or the fact that the stochastic upper limits are very high.

%\begin{align}
%\#_{\text{shell}} &= N(z)dz \nolabel \\
%&= N(r)dr , \label{eq:z_r_param}
%\end{align}
%where $r$ is the comoving radius of the spherical shell. 
%We then use this expression to replace $N(z)dz$ in Equation \ref{eq:phinney_theorem}, obtaining 
%
%\begin{equation*}
%\rho_c c^2 \Ogw(f) =  \int_{0}^{\infty} \frac{N(r)}{1+z} (f_r \frac{dE_{\mathrm{gw}}}{df_r})_{f_r = f(1+z)} \mathrm{d}r .
%\end{equation*}
\bibliography{euclidian_calc_notes}{}
\bibliographystyle{ieeetr}

\end{document}
